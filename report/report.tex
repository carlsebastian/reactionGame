\documentclass[a4paper]{article}

%% Language and font encodings
\usepackage[english]{babel}
\usepackage[utf8x]{inputenc}
\usepackage[T1]{fontenc}
\usepackage{mathtools}
\usepackage{graphicx}
\usepackage{subcaption}
\usepackage{float}
\usepackage{a4wide}
\usepackage{amsmath}
\usepackage{rotating}

%% Sets page size and margins
\usepackage[a4paper,top=3cm,bottom=2cm,left=3cm,right=3cm,marginparwidth=1.75cm]{geometry}

\title{DataCom Project}
\author{Alexander Backlund, Sebastian Gustafsson, Siri Göhl, Albin Hjelm}

\begin{document}
\maketitle
\newpage

\section{Project idé}
\subsection{Förslag}
Ett spel där man tävlar 1 mot 1 i reaktionshastighet. Spelarna delar en gemensam
spelplan som när spelet startar är tom. En figur av något slag slumpas fram på
spelplanen och det gäller nu att vara den första som hinner klicka på figuren.
När det är avgjort vem som först klickat på figuren fösvinner den från spelplanen och
dyker upp under den spelares namn som klickade först. Nu är spelplanen återigen tom
och nästa figur slumpas fram och processen återupprepas. Först den till x antal vunna
figurer har vunnit.

\subsection{Delad resurs}
Den delade resursen i spelet är den slumpade figuren som bara en kan få tillgång till
och vi kommer behöva en metod av någon form som kan avgöra vilken spelare som
klickade först. Sedan skriver man till ”permanent storage” med information om
poängställning, vems tur, mm.

\subsection{Uppdatering av globala tillstånd}
Här måste spelplanen och resultaten uppdateras i realtid för båda spelarna.

\subsection{Strategier för eventuella problem}
Vi måste kunna:
\begin{itemize}
\item avgöra vem som klickat först.
\item hantera situationen då två spelare trycker exakt samtidigt.
\item hantera anslutningsproblem.
\item använda lås för att undvika korrumperad data.
\end{itemize}

\subsection{Testning}
Använda oss av minst tre datorer varav två klienter och en host.

\subsection{Designförslag}
\begin{figure}[H]
\includegraphics[width=0.8\textwidth]{designf_rslag.jpeg}
\caption{Design förslag}
\end{figure}

Vi kommer implementera systemet i Python, samt använda sockets för kommunikation mellan
systemen.

\subsection{Demonstration av systemet}
Demonstrationen kommer att fokusera på kommunikationen mellan servern och klienten.

\section{System design}
\subsection{Övergripande system design}
\subsubsection{Delar i systemet}
\begin{itemize}
\item klienter
\item Host/server
\item Permanent storage
\end{itemize}

\subsubsection{Hur delarna interagerar med varandra}

\begin{figure}[H]
\includegraphics[width=0.8\textwidth]{design.png}
\caption{Hur systemets delar interagerar med varandra}
\end{figure}

\subsubsection{Programmerings miljö}
Vi kommer använda oss av Python och deras socketpaket. Vi använder socketpaketet för att kunna etablera nätverkskanaler mellan servern och klienter. Vi kommer även använda ett grafiskt paket som vi ännu inte har bestämt.
\subsubsection{Problem som kan uppstå}
Vi måste kunna:
\begin{itemize}
\item avgöra vem som klickat först - Tidstämpel hos klienterna.
\item hantera situationen då två spelare trycker exakt samtidigt -  Tidstämpel hos klienterna.
\item hantera om en klients anslutning bryts - Servern hanterar att klienten har förlorat kontakten genom att avsluta spelet och sparar data till permanentat storage.
\item hantera om servern går ner - Servern loggar vinster, förluster. start och avslut av spel. När servern går ner startas servern om och läser av loggen. Om det senaste händelsen inte är avslutat spel då läser servern in ovanstående korrekta händelser.
\item uppnå Kausal ordning - Uppnås genom att använda tidstämplar på kommunikationen från klienterna till servern.
\item kontrollera att tidstämpeln är korrekt -
\end{itemize}


\subsection{Preliminär tidsplan}
\subsubsection{Vad kommer varje medlem göra?}
Vi kommer arbeta i par varav ena paret kommer fokusera på gränssnittet och andra paret kommer att fokusera på server-klient kommunikation. Detta gör vi en vecka åt gången sedan kommer vi rotera i paren. Beroende på tidsåtgång och arbetsbörda får vi efter varje arbetsvecka eventuellt modifiera arbetsuppgifterna för de två grupperna.
\subsubsection{Hur långt kommer vi hinna innan 30:de november?}
Vi försöker sätta upp veckomål, hur lång vi vill ha kommit efter varje vecka. Fram tom. den 30 november vill vi ha et färdigt gränssnitt samt en fungerande klient server kommunikation med de krav tidsstämpling osv. som vi anser viktiga i vår nuvarande bild av systemet. Förhoppningsvis har vi hunnit påbörja integreringen mellan gränssnitt och kommunikationen.

\newpage

\section{Slutgiltiga reflektioner}

\subsection{Vad kunde vi ha gjort bättre?}
Under redovisningen insåg vi att programmet kraschar när spelare försöker logga in med icke-engelska bokstäver. Ifall en spelare har ett inlogg med svenska bokstäver som Å,Ä eller Ö, så orsakade detta en krasch för hela programmet. Därför vore det en prioriterad vidareutveckling att implementera stöd för andra bokstäver än engelska. \\
\\
Ett annat relativt allvarligt problem, som även det kan orska en krash, är att programmet är väldigt sekvensiellt. Exempelvis så har vi inte något stöd för försvunna packet. Stöd för försvunna packet skulle kunna vara återsändning av dessa paket eller alternativt att använda TCP protocol istället för UDP i vissa fall. \\
\\
Den nuvarande versionen av spelet har endast trianglar, cirklar och fyrkanter. Dessa figurer är rätt enkla att både skapa och att räkna på. Därför skulle det vara en rolig utmaning att implementera en generell formel för n-hörningar. Vi har redan en bra algoritm för att räkna ut ifall en spelare har klickat innanför en triangel eller inte. Därför vore det lämpligt att använda sig av den redan befintliga triangel-algoritm och försöka göra den generell för n-hörningar\\
\\

\subsection{Hur har arbetet fortlöpt?}
Vi har arbetat i par, ett front-end och ett back-end par. Det har fungerat bra att arbeta med denna uppdelning. Kommunikationen mellan grupparna har fungerat mycket bra. Vid ett par tillfällen så har back-end gruppen behövt specifika funktioner från front-end gruppen för tester. Eftersom kommunikationen har fungerat så pass bra som den har så har back-end gruppen aldrig hamnat i en situation där de inte har kunna arbeta. \\
\\

\section{Slutgiltig design}
Ett spel där man tävlar 1 mot N antal spelare i reaktionshastighet. Spelarna delar en gemensam spelplan som är tom när spelet startar. En kvadrat, en cirkel eller en triangel slumpas fram på spelplanen och det gäller att vara den första som klickar på figuren. Servern skickar vilken tid som figuren ska visas för spelarna. Här har vi lagt en slumpad delay på  1-5 sekunder. När en spelare klickar på figuren registreras den aktuella tiden minus tiden då figuren visades. Denna differens skickas sedan till servern som väljer den minsta tiden som den vinnande. Spelaren som skickat den tiden får ett poäng.


\end{document}
